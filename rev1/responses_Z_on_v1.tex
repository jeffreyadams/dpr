(15) $D$ was defined in the second paragraph of 2.5

(23) I don't understand the suggested correction, because it is the same as what is written now. Perhaps a copy-past mistake?

(24) This is the notation used in [Kal16b]. It is meant to suggest taking the preimage in $\hat{\bar H}$ or $\hat{\bar G}$ of a set of Galois-fixed points. For this we think of $S_\vaprhi$ as Galois-fixed points in $\hat G$ for the Galois action given by $\varphi$.

(32) The Lie group $N(\R)$ is a nilpotent group, a condition which is used in the cited theorem.

(38) I guess it's a matter of taste. I'm allowing an arbitrary fourth root of unity.

(44) $A$ contains elements outside of $\hat G$, while $B$ lies in $\hat G$, so $B$ never equals $A$.
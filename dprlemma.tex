\documentclass{article}
\usepackage {amsmath, amsfonts, amssymb, mathrsfs, amsthm, sectsty,hyphenat,enumitem,calc,url,color,array,tabu}
%\usepackage{showlabels}
\usepackage{palatino}
\usepackage[all]{xy}
\usepackage[nottoc]{tocbibind}
\usepackage[toc,page]{appendix}
\usepackage{tocloft}

\renewcommand\cftsecafterpnum{\vskip2pt}
\renewcommand\cftsubsecafterpnum{\vskip2pt}


\newcommand{\redwarn}[1]{\textcolor{red}{#1}\message{#1}}


\setlength{\parindent}{0pt} \setlength{\parskip}{10pt plus 2pt minus 1pt}
\topmargin=-0.5in \headheight=0in \headsep=0.25in \textheight=9.1in
\footskip=0.75in

%\def\mf#1{\mathfrak{#1}}
\def\mc#1{\mathcal{#1}}
\def\mb#1{\mathbb{#1}}
\def\tx#1{\textrm{#1}}
\def\tb#1{\textbf{#1}}
\def\ti#1{\textit{#1}}
\def\ts#1{\textsf{#1}}
\def\ms#1{\mathsf{#1}}
\def\tr{\tx{tr}\,}
\def\R{\mathbb{R}}
\def\P{\mathbb{P}}
\def\C{\mathbb{C}}
\def\Q{\mathbb{Q}}
\def\A{\mathbb{A}}
\def\Z{\mathbb{Z}}
\def\N{\mathbb{N}}
\def\F{\mathbb{F}}
\def\lmod{\backslash}
\def\Gal{\rm{Gal}}
\def\ol#1{\overline{#1}}
\def\ul#1{\underline{#1}}
\def\sp{\tx{Spec}}
\def\rk{\tx{rk}}
\def\Ad{\tx{Ad}}
\def\hat{\widehat}
\def\vol{\tx{vol}}
\def\conj#1{\ \underaccent{#1}{\sim}\ }
\def\rw{\rightarrow}
\def\lw{\leftarrow}
\def\from{\leftarrow}
\def\dw{\leftrightarrow}
\def\lrw{\longrightarrow}
\def\llw{\longleftarrow}
\def\hrw{\hookrightarrow}
\def\irw{\hookrightarrow}
\def\into{\hookrightarrow}
\def\hlw{\hookleftarrow}
\def\ilw{\hookleftarrow}
\def\thrw{\twoheadrightarrow}
\def\srw{\twoheadrightarrow}
\def\onto{\twoheadrightarrow}
\def\rrw{\rightrightarrows}
\def\lw{\leftarrow}
\def\sm{\smallsetminus}
\def\la{\langle}
\def\ra{\rangle}
\def\<{\langle}
\def\>{\rangle}
\def\Wedge{\bigwedge}
\def\mr#1{\mathring{#1}}

%\newarrow{Equals}{=}{=}{=}{=}{}

\def\rmk{\tb{Remark: }}
\def\ex{\tb{Example: }}
\def\pf{\tb{Proof: }}
\def\nt{\tb{Note: }}

\newenvironment{mytitle}
{\begin{center}\large\sc}
{\end{center}}

\def\beginappendix{ \newpage\renewcommand{\thesection}{}\section{Appendix}\setcounter{subsection}{0}\renewcommand{\thesubsection}{\Alph{subsection}}}
\def\appsection#1{ \subsection{#1}}


\hyphenation{ar-chi-me-de-an}
\hyphenation{an-iso-tro-pic}
%\def\mf#1{\mathfrak{#1}}
\def\mc#1{\mathcal{#1}}
\def\mb#1{\mathbb{#1}}
\def\tx#1{\textrm{#1}}
\def\tb#1{\textbf{#1}}
\def\ti#1{\textit{#1}}
\def\ts#1{\textsf{#1}}
\def\ms#1{\mathsf{#1}}
\def\tr{\tx{tr}\,}
\def\R{\mathbb{R}}
\def\P{\mathbb{P}}
\def\C{\mathbb{C}}
\def\Q{\mathbb{Q}}
\def\A{\mathbb{A}}
\def\Z{\mathbb{Z}}
\def\N{\mathbb{N}}
\def\F{\mathbb{F}}
\def\lmod{\backslash}
\def\Gal{\rm{Gal}}
\def\ol#1{\overline{#1}}
\def\ul#1{\underline{#1}}
\def\sp{\tx{Spec}}
\def\rk{\tx{rk}}
\def\Ad{\tx{Ad}}
\def\hat{\widehat}
\def\vol{\tx{vol}}
\def\conj#1{\ \underaccent{#1}{\sim}\ }
\def\rw{\rightarrow}
\def\lw{\leftarrow}
\def\from{\leftarrow}
\def\dw{\leftrightarrow}
\def\lrw{\longrightarrow}
\def\llw{\longleftarrow}
\def\hrw{\hookrightarrow}
\def\irw{\hookrightarrow}
\def\into{\hookrightarrow}
\def\hlw{\hookleftarrow}
\def\ilw{\hookleftarrow}
\def\thrw{\twoheadrightarrow}
\def\srw{\twoheadrightarrow}
\def\onto{\twoheadrightarrow}
\def\rrw{\rightrightarrows}
\def\lw{\leftarrow}
\def\sm{\smallsetminus}
\def\la{\langle}
\def\ra{\rangle}
\def\<{\langle}
\def\>{\rangle}
\def\Wedge{\bigwedge}
\def\mr#1{\mathring{#1}}

%\newarrow{Equals}{=}{=}{=}{=}{}

\def\rmk{\tb{Remark: }}
\def\ex{\tb{Example: }}
\def\pf{\tb{Proof: }}
\def\nt{\tb{Note: }}

\newenvironment{mytitle}
{\begin{center}\large\sc}
{\end{center}}

\def\beginappendix{ \newpage\renewcommand{\thesection}{}\section{Appendix}\setcounter{subsection}{0}\renewcommand{\thesubsection}{\Alph{subsection}}}
\def\appsection#1{ \subsection{#1}}


\hyphenation{ar-chi-me-de-an}
\hyphenation{an-iso-tro-pic}
\def\mf#1{\mathfrak{#1}}
\def\mc#1{\mathcal{#1}}
\def\mb#1{\mathbb{#1}}
\def\tx#1{\textrm{#1}}
\def\tb#1{\textbf{#1}}
\def\ti#1{\textit{#1}}
\def\ts#1{\textsf{#1}}
\def\ms#1{\mathsf{#1}}
\def\tr{\tx{tr}\,}
\def\R{\mathbb{R}}
\def\P{\mathbb{P}}
\def\C{\mathbb{C}}
\def\Q{\mathbb{Q}}
\def\A{\mathbb{A}}
\def\Z{\mathbb{Z}}
\def\N{\mathbb{N}}
\def\F{\mathbb{F}}
\def\lmod{\backslash}
\def\Gal{\rm{Gal}}
\def\ol#1{\overline{#1}}
\def\ul#1{\underline{#1}}
\def\sp{\tx{Spec}}
\def\rk{\tx{rk}}
\def\Ad{\tx{Ad}}
\def\hat{\widehat}
\def\vol{\tx{vol}}
\def\conj#1{\ \underaccent{#1}{\sim}\ }
\def\rw{\rightarrow}
\def\lw{\leftarrow}
\def\from{\leftarrow}
\def\dw{\leftrightarrow}
\def\lrw{\longrightarrow}
\def\llw{\longleftarrow}
\def\hrw{\hookrightarrow}
\def\irw{\hookrightarrow}
\def\into{\hookrightarrow}
\def\hlw{\hookleftarrow}
\def\ilw{\hookleftarrow}
\def\thrw{\twoheadrightarrow}
\def\srw{\twoheadrightarrow}
\def\onto{\twoheadrightarrow}
\def\rrw{\rightrightarrows}
\def\lw{\leftarrow}
\def\sm{\smallsetminus}
\def\la{\langle}
\def\ra{\rangle}
\def\<{\langle}
\def\>{\rangle}
\def\Wedge{\bigwedge}
\def\mr#1{\mathring{#1}}

%\newarrow{Equals}{=}{=}{=}{=}{}

\def\rmk{\tb{Remark: }}
\def\ex{\tb{Example: }}
\def\pf{\tb{Proof: }}
\def\nt{\tb{Note: }}

\newenvironment{mytitle}
{\begin{center}\large\sc}
{\end{center}}

\def\beginappendix{ \newpage\renewcommand{\thesection}{}\section{Appendix}\setcounter{subsection}{0}\renewcommand{\thesubsection}{\Alph{subsection}}}
\def\appsection#1{ \subsection{#1}}


\hyphenation{ar-chi-me-de-an}
\hyphenation{an-iso-tro-pic}


\newtheorem{thm}{Theorem}[subsection]
\newtheorem{lem}[thm]{Lemma}
\newtheorem{pro}[thm]{Proposition}
\newtheorem{cor}[thm]{Corollary}
\newtheorem{fct}[thm]{Fact}
\newtheorem{clm}[thm]{Claim}
\newtheorem{asm}[thm]{Assumption}
\newtheorem{cnj}[thm]{Conjecture}
\newtheorem{exn}[thm]{Expectation}
\newtheorem{ntt}[thm]{Notation}
\theoremstyle{definition}
\newtheorem{dfn}[thm]{Definition}
\newtheorem{rem}[thm]{Remark}
\newtheorem{exa}[thm]{Example}
\newtheorem{cns}[thm]{Construction}

\sectionfont{\center\sc\normalsize}
\subsectionfont{\bf\normalsize}

\numberwithin{equation}{section}
\renewcommand{\-}{\hyp{}}
\hyphenation{co-cycle co-chain co-ho-mo-lo-gy}

\renewcommand{\appendixtocname}{Appendix}
\renewcommand{\appendixpagename}{\normalsize \center Appendix}

\newcommand{\warn}[1]{{\leavevmode\color{red}[#1]}}
%definitions by jda
\newcommand{\inv}{^{-1}}
\newcommand{\z}{\mathfrak Z}
\newcommand{\U}{\mathfrak U}
\newcommand{\n}{\mathfrak n}
\newcommand{\s}{\mathfrak s}
\newcommand{\g}{\mathfrak g}
\newcommand{\p}{\mathfrak p}
\renewcommand{\k}{\mathfrak k}
\renewcommand{\O}{\mathcal O}
\renewcommand{\P}{\mathcal P}
\newcommand{\K}{\mathcal K}
\newcommand{\Op}{\O_p}
\newcommand{\Oss}{\O_{ss}}
\newcommand{\wx}{\mathfrak w_X}
\newcommand{\w}{\mathfrak w}
\newcommand{\h}{\mathfrak h}
\newcommand{\Cent}{\mathrm{Cent}}
\newcommand{\ad}{\mathrm{ad}}
\newcommand{\WF}{\mathrm{WF}}
\newcommand{\AV}{\mathrm{AV}}
\newcommand{\AS}{\mathrm{AS}}
\newcommand{\KS}{\mathrm{KS}}
\newcommand{\Cone}{\mathrm{Cone}}
\newcommand{\AC}{\mathrm{AC}}
\newcommand{\SL}{\mathrm{SL}}
\newcommand{\ch}[1]{\negthinspace\negthinspace\negthinspace\phantom{a}^\vee\negthinspace #1}

\begin{document}

\begin{mytitle} A Result of Harish Chandra\end{mytitle}


Let $G$ be
 a connected real reductive group.
We let $\g=\mathrm{Lie}(G)$ and $\g_\C=\g\otimes\C$.
Let $G_r$ be the open set of regular semisimple elements of $G$. Suppose $H$ is a  Cartan subgroup of $G$, and set $\h=\mathrm{Lie}(H)\otimes\C$ and 
$H_r=H\cap G_r$.
Define
$$
\Delta(h)=(\det(\Ad(h)\inv -1)|_{\g/\h}\quad (a,h\in H_r).
$$
For later use we also define
$$
\Delta_a(h)=\Delta(ah)=(\det(\Ad(ah)\inv -1)|_{\g/\h}\quad (h\in H_r).
$$

Let $\U(\g)$ be the universal enveloping algebra. 
An element $X\in \U(\g)$ defines a left-invariant differential operator $\tau(X)$.
For $x\in \g$ the definition is
$$\tau(X)(f)(g)=\frac d{dt}g\exp(tX)|_{t=0}.
$$
Let $L_g(f)(h)=f(gh)$ $(g,h\in G,f\in C^\infty(G))$. A differential operator $D$ being left-invariant means 
\begin{equation}
\label{e:left}
D(L_gf)(h)=D(f)(gh).
\end{equation}

It is customary to drop the $\tau$ notation, and write $Xf$ instead of $\tau(X)f$. I prefer to keep it for reasons that should become clear.
Let $\z$ be the center of  $\U(\g)$.
Let $W$ be the Weyl group of $H$ in $G$, and 
let $\gamma:\z\rightarrow \U(\h)^W$ be the Harish Chandra isomorphism.
Let $G_r$ be the open set of regular semisimple elements of $G$, and let $C^\infty$ be the smooth function on $G_r$. 


\begin{pro}
Suppose $f\in C^\infty(G_r)$ is a class function. Then
$$
(\tau(z)f)(h)=|\Delta(h)|^{-\frac12}\tau(\gamma(z))(|\Delta|^{\frac12}f)(h)\quad (h\in H_r).
$$
\end{pro}

A less pedantic way of writing this is to drop $\tau$ and identify $\U(\g)$ with the left-invariant differential operators on $G$. Then
the conclusion is
$$
(zf)=|\Delta(h)|^{-\frac12}\gamma(z)(|\Delta|^{\frac12}f)
$$
as functions on $H_r$.
Before giving the proof we establish some notation and preliminary results.



Suppose $D$ is a differential operator on $G$. Then $D$ has a local expression $D_g\in\U(\g)$ for any $g\in G$.
See \cite[Section 4]{HC_characters}.
By definition this means
\begin{equation}
  \label{e:tau}
  D(f)(g)=\tau(D_g)(f)(g)\quad(g\in G)
\end{equation}

Note that $D$ is left-invariant if and only if
\begin{equation}
  \label{e:left}
  D_g=D_h\quad(\text{for all }g,h\in G)
\end{equation}
To see this, note that for $h,g\in G$, left-invariance of $D$ says:
$$
D(f)(hg)=D(L_hf)(g).
$$
The left hand side is $\tau(D_{hg})(f)(hg)$. The right hand side is
$$
\tau(D_g)(L_h(f))(g)=\tau(D_g)(f)(hg)
$$
since $\tau(D_g)$ is (obviously)  left-invariant. Therefore $\tau(D_{hg})=\tau(D_g)$, and therefore, $D_{hg}=D_g$ for all $g,h$.


Suppose $a\in G$ is a regular semisimple element, and set $H=\Cent_G(a)$. Suppose $U$ is an open set in $G$ containing $a$, 
and set
\begin{equation}
  \label{e:UH}
U_H=a\inv U\cap H_r.
\end{equation}
This is an open neighborhood of $1$ in $H_r$.
Suppose $D$ is a differential operator on $U$. Define $\delta_a(D)$, a differential operator  on $U_H$ by \cite[Section 4]{HC_inv_eigen}.
Note that if $y\in U_H$ then $\delta_a(D)_y\in \U(\h)$. 

See \cite[Section 8]{HC_inv_dist} for the definition of locally invariant functions.
The restriction of a class function on $G_r$ to any open set is locally invariant.

\begin{lem}
\label{l:1}
Suppose $a,H,U$ and $U_H$ are as above. 
Suppose $D$ is a differential operator on $U$, and $f$ is a locally invariant function on $U$.
Then for $y\in U_H$:
$$
D(f)(ay)=\tau(\delta_a(D)_y)(f)(ay)
$$
\end{lem}

This is \cite[Lemma 18]{HC_inv_eigen}. Some care is required here.

The statement of \cite[Lemma 18]{HC_inv_eigen} says
$$
D(f)(ay)=\delta_a(D)(f)(ay)
$$
However the right hand side isn't defined: $\delta_a(D)$ is a differential operator on $U_H$, but $ay$ is in $U$, but not $U_H$.
What the proof actually shows (see the last line of the proof) is that 
$$
\tau(D_{ay})(f)(ay)=\tau(\delta_a(D)_y)(f)(ay).
$$
Note that $\tau(D_{ay})$ is a well defined left-invariant differential operator on $U_H$.
By \eqref{e:tau}
the LHS equals $D(f)(ay)$, so this gives the statement of the Lemma.

\begin{cor}
  Suppose $f$ is a class function on $G_r$ and $D$ is a left-invariant differential operator on $G_r$.
  Suppose $a\in G_r$ is semisimple and set $H=\Cent_G(a)$. Then $U_H=H_r$ (see \eqref{e:UH}), so $\delta_a(D)$
  is a  differential operator on $H_r$.

  Assume $\delta_a(D)$ is left-invariant on $H_r$. Then
  $$
  D(f)(h)=\delta_a(D)(f)(h)\quad (h\in H_r)
  $$
\end{cor}

\begin{rem}
I think $D$ is automatically left-invariant on $H_r$, but I'm not sure. In any event we don't use the Corollary.
\end{rem}

\begin{proof}
  From the Lemma we have
  $$
  D(f)(ay)=\tau(\delta_a(D)_y)(f)(ay)
  $$
  for all $y\in U_H=H_r$. Since $\delta_a(D)$ is left-invariant so by \eqref{e:left}
  $$
  \tau(\delta_a(D)_y)(f)(ay)=  \tau(\delta_a(D)_{ay})(f)(ay)=\delta_a(D)(ay)
  $$
\end{proof}

\begin{lem}
\label{l:2}
$$
\delta_a(\tau(z))(f)(h)=|\Delta_a(h)|^{-\frac12}\tau(\gamma(z))(|\Delta_a|^{\frac12}f)(h)
$$
\end{lem}
This is \cite[Lemma 13]{HC_inv_eigen}.

\begin{proof}[Proof of the Proposition]

Let $U$ be an open set containing $h$ and let $U_H=h\inv U\cap H_r$ as before. 
Write $h=ay$ where $a\in H_r$ and $y\in U_H$.
Suppose $z\in\z(\g)$. Then:

$$
\begin{aligned}
  \tau(z)(f)(ay)&=\tau(\delta_a(\tau(z))_y)(f)(ay)\quad\text{(by Lemma \ref{l:2})}\\
&=  \tau(\delta_a(\tau(z))_y)(L_a(f))(y)\\
 &= \delta_a(\tau(z))(L_a(f))(y)\quad\text{(definition of $L_a$)}\\
 &= |\Delta_a(y)|^{-\frac12}\tau(\gamma(z))(|\Delta_a|^{\frac12}L_a(f))(y)\quad\text{(by Lemma \ref{l:2})}\\
 &= |\Delta_a(y)|^{-\frac12}\tau(\gamma(z))(L_a(|\Delta|^{\frac12}f))(y)\quad\text{(elementary)}\\
   &= |\Delta_a(y)|^{-\frac12}\tau(\gamma(z))(|\Delta|^{\frac12}f))(ay)\quad\text{(by left-invariance of $\tau(\gamma(z))$)}
\end{aligned}
$$
The elementary step is the equality of the functions $|\Delta_a|^{\frac12}L_a(f)$ 
and
 $L_a(|\Delta|^{\frac12}f)$.

This proves the Proposition.
\end{proof}




\bibliographystyle{amsalpha}
%\bibliography{/Users/kaletha/Work/TexMain/bibliography.bib}
\bibliography{bibliography.bib}

\end{document}


\documentclass{article}
\usepackage {amsmath, amsfonts, amssymb, mathrsfs, amsthm, sectsty,hyphenat,enumitem,calc,url,color,array,tabu}
%\usepackage{showlabels}
\usepackage{palatino}
\usepackage[all]{xy}
\usepackage[nottoc]{tocbibind}
\usepackage[toc,page]{appendix}
\usepackage{tocloft}

\renewcommand\cftsecafterpnum{\vskip2pt}
\renewcommand\cftsubsecafterpnum{\vskip2pt}


\newcommand{\redwarn}[1]{\textcolor{red}{#1}\message{#1}}


\setlength{\parindent}{0pt} \setlength{\parskip}{10pt plus 2pt minus 1pt}
\topmargin=-0.5in \headheight=0in \headsep=0.25in \textheight=9.1in
\footskip=0.75in

%\def\mf#1{\mathfrak{#1}}
\def\mc#1{\mathcal{#1}}
\def\mb#1{\mathbb{#1}}
\def\tx#1{\textrm{#1}}
\def\tb#1{\textbf{#1}}
\def\ti#1{\textit{#1}}
\def\ts#1{\textsf{#1}}
\def\ms#1{\mathsf{#1}}
\def\tr{\tx{tr}\,}
\def\R{\mathbb{R}}
\def\P{\mathbb{P}}
\def\C{\mathbb{C}}
\def\Q{\mathbb{Q}}
\def\A{\mathbb{A}}
\def\Z{\mathbb{Z}}
\def\N{\mathbb{N}}
\def\F{\mathbb{F}}
\def\lmod{\backslash}
\def\Gal{\rm{Gal}}
\def\ol#1{\overline{#1}}
\def\ul#1{\underline{#1}}
\def\sp{\tx{Spec}}
\def\rk{\tx{rk}}
\def\Ad{\tx{Ad}}
\def\hat{\widehat}
\def\vol{\tx{vol}}
\def\conj#1{\ \underaccent{#1}{\sim}\ }
\def\rw{\rightarrow}
\def\lw{\leftarrow}
\def\from{\leftarrow}
\def\dw{\leftrightarrow}
\def\lrw{\longrightarrow}
\def\llw{\longleftarrow}
\def\hrw{\hookrightarrow}
\def\irw{\hookrightarrow}
\def\into{\hookrightarrow}
\def\hlw{\hookleftarrow}
\def\ilw{\hookleftarrow}
\def\thrw{\twoheadrightarrow}
\def\srw{\twoheadrightarrow}
\def\onto{\twoheadrightarrow}
\def\rrw{\rightrightarrows}
\def\lw{\leftarrow}
\def\sm{\smallsetminus}
\def\la{\langle}
\def\ra{\rangle}
\def\<{\langle}
\def\>{\rangle}
\def\Wedge{\bigwedge}
\def\mr#1{\mathring{#1}}

%\newarrow{Equals}{=}{=}{=}{=}{}

\def\rmk{\tb{Remark: }}
\def\ex{\tb{Example: }}
\def\pf{\tb{Proof: }}
\def\nt{\tb{Note: }}

\newenvironment{mytitle}
{\begin{center}\large\sc}
{\end{center}}

\def\beginappendix{ \newpage\renewcommand{\thesection}{}\section{Appendix}\setcounter{subsection}{0}\renewcommand{\thesubsection}{\Alph{subsection}}}
\def\appsection#1{ \subsection{#1}}


\hyphenation{ar-chi-me-de-an}
\hyphenation{an-iso-tro-pic}
%\def\mf#1{\mathfrak{#1}}
\def\mc#1{\mathcal{#1}}
\def\mb#1{\mathbb{#1}}
\def\tx#1{\textrm{#1}}
\def\tb#1{\textbf{#1}}
\def\ti#1{\textit{#1}}
\def\ts#1{\textsf{#1}}
\def\ms#1{\mathsf{#1}}
\def\tr{\tx{tr}\,}
\def\R{\mathbb{R}}
\def\P{\mathbb{P}}
\def\C{\mathbb{C}}
\def\Q{\mathbb{Q}}
\def\A{\mathbb{A}}
\def\Z{\mathbb{Z}}
\def\N{\mathbb{N}}
\def\F{\mathbb{F}}
\def\lmod{\backslash}
\def\Gal{\rm{Gal}}
\def\ol#1{\overline{#1}}
\def\ul#1{\underline{#1}}
\def\sp{\tx{Spec}}
\def\rk{\tx{rk}}
\def\Ad{\tx{Ad}}
\def\hat{\widehat}
\def\vol{\tx{vol}}
\def\conj#1{\ \underaccent{#1}{\sim}\ }
\def\rw{\rightarrow}
\def\lw{\leftarrow}
\def\from{\leftarrow}
\def\dw{\leftrightarrow}
\def\lrw{\longrightarrow}
\def\llw{\longleftarrow}
\def\hrw{\hookrightarrow}
\def\irw{\hookrightarrow}
\def\into{\hookrightarrow}
\def\hlw{\hookleftarrow}
\def\ilw{\hookleftarrow}
\def\thrw{\twoheadrightarrow}
\def\srw{\twoheadrightarrow}
\def\onto{\twoheadrightarrow}
\def\rrw{\rightrightarrows}
\def\lw{\leftarrow}
\def\sm{\smallsetminus}
\def\la{\langle}
\def\ra{\rangle}
\def\<{\langle}
\def\>{\rangle}
\def\Wedge{\bigwedge}
\def\mr#1{\mathring{#1}}

%\newarrow{Equals}{=}{=}{=}{=}{}

\def\rmk{\tb{Remark: }}
\def\ex{\tb{Example: }}
\def\pf{\tb{Proof: }}
\def\nt{\tb{Note: }}

\newenvironment{mytitle}
{\begin{center}\large\sc}
{\end{center}}

\def\beginappendix{ \newpage\renewcommand{\thesection}{}\section{Appendix}\setcounter{subsection}{0}\renewcommand{\thesubsection}{\Alph{subsection}}}
\def\appsection#1{ \subsection{#1}}


\hyphenation{ar-chi-me-de-an}
\hyphenation{an-iso-tro-pic}
\def\mf#1{\mathfrak{#1}}
\def\mc#1{\mathcal{#1}}
\def\mb#1{\mathbb{#1}}
\def\tx#1{\textrm{#1}}
\def\tb#1{\textbf{#1}}
\def\ti#1{\textit{#1}}
\def\ts#1{\textsf{#1}}
\def\ms#1{\mathsf{#1}}
\def\tr{\tx{tr}\,}
\def\R{\mathbb{R}}
\def\P{\mathbb{P}}
\def\C{\mathbb{C}}
\def\Q{\mathbb{Q}}
\def\A{\mathbb{A}}
\def\Z{\mathbb{Z}}
\def\N{\mathbb{N}}
\def\F{\mathbb{F}}
\def\lmod{\backslash}
\def\Gal{\rm{Gal}}
\def\ol#1{\overline{#1}}
\def\ul#1{\underline{#1}}
\def\sp{\tx{Spec}}
\def\rk{\tx{rk}}
\def\Ad{\tx{Ad}}
\def\hat{\widehat}
\def\vol{\tx{vol}}
\def\conj#1{\ \underaccent{#1}{\sim}\ }
\def\rw{\rightarrow}
\def\lw{\leftarrow}
\def\from{\leftarrow}
\def\dw{\leftrightarrow}
\def\lrw{\longrightarrow}
\def\llw{\longleftarrow}
\def\hrw{\hookrightarrow}
\def\irw{\hookrightarrow}
\def\into{\hookrightarrow}
\def\hlw{\hookleftarrow}
\def\ilw{\hookleftarrow}
\def\thrw{\twoheadrightarrow}
\def\srw{\twoheadrightarrow}
\def\onto{\twoheadrightarrow}
\def\rrw{\rightrightarrows}
\def\lw{\leftarrow}
\def\sm{\smallsetminus}
\def\la{\langle}
\def\ra{\rangle}
\def\<{\langle}
\def\>{\rangle}
\def\Wedge{\bigwedge}
\def\mr#1{\mathring{#1}}

%\newarrow{Equals}{=}{=}{=}{=}{}

\def\rmk{\tb{Remark: }}
\def\ex{\tb{Example: }}
\def\pf{\tb{Proof: }}
\def\nt{\tb{Note: }}

\newenvironment{mytitle}
{\begin{center}\large\sc}
{\end{center}}

\def\beginappendix{ \newpage\renewcommand{\thesection}{}\section{Appendix}\setcounter{subsection}{0}\renewcommand{\thesubsection}{\Alph{subsection}}}
\def\appsection#1{ \subsection{#1}}


\hyphenation{ar-chi-me-de-an}
\hyphenation{an-iso-tro-pic}


\newtheorem{thm}{Theorem}[subsection]
\newtheorem{lem}[thm]{Lemma}
\newtheorem{pro}[thm]{Proposition}
\newtheorem{cor}[thm]{Corollary}
\newtheorem{fct}[thm]{Fact}
\newtheorem{clm}[thm]{Claim}
\newtheorem{asm}[thm]{Assumption}
\newtheorem{cnj}[thm]{Conjecture}
\newtheorem{exn}[thm]{Expectation}
\newtheorem{ntt}[thm]{Notation}
\theoremstyle{definition}
\newtheorem{dfn}[thm]{Definition}
\newtheorem{rem}[thm]{Remark}
\newtheorem{exa}[thm]{Example}
\newtheorem{cns}[thm]{Construction}

\sectionfont{\center\sc\normalsize}
\subsectionfont{\bf\normalsize}

\numberwithin{equation}{section}
\renewcommand{\-}{\hyp{}}
\hyphenation{co-cycle co-chain co-ho-mo-lo-gy}

\renewcommand{\appendixtocname}{Appendix}
\renewcommand{\appendixpagename}{\normalsize \center Appendix}

\newcommand{\warn}[1]{{\leavevmode\color{red}[#1]}}
%definitions by jda
\newcommand{\inv}{^{-1}}
\newcommand{\z}{\mathfrak Z}
\newcommand{\n}{\mathfrak n}
\newcommand{\s}{\mathfrak s}
\newcommand{\g}{\mathfrak g}
\newcommand{\p}{\mathfrak p}
\renewcommand{\k}{\mathfrak k}
\renewcommand{\O}{\mathcal O}
\renewcommand{\P}{\mathcal P}
\newcommand{\K}{\mathcal K}
\newcommand{\Op}{\O_p}
\newcommand{\Oss}{\O_{ss}}
\newcommand{\wx}{\mathfrak w_X}
\newcommand{\w}{\mathfrak w}
\newcommand{\h}{\mathfrak h}
\newcommand{\Cent}{\mathrm{Cent}}
\newcommand{\ad}{\mathrm{ad}}
\newcommand{\WF}{\mathrm{WF}}
\newcommand{\AV}{\mathrm{AV}}
\newcommand{\AS}{\mathrm{AS}}
\newcommand{\KS}{\mathrm{KS}}
\newcommand{\Cone}{\mathrm{Cone}}
\newcommand{\AC}{\mathrm{AC}}
\newcommand{\SL}{\mathrm{SL}}
\newcommand{\ch}[1]{\negthinspace\negthinspace\negthinspace\phantom{a}^\vee\negthinspace #1}

\begin{document}

\begin{mytitle} A Result of Harish Chandra\end{mytitle}


Let $G$ be a connected real reductive group.
We let $\g=\mathrm{Lie}(G)$ and $\g_\C=\g\otimes\C$.
Let $\theta$ be a Cartan involution of $G$.

Let $G_r$ be the open set of regular semisimple elements if $G$. Suppose $H$ is a $\theta$-stable Cartan subgroup of $G$, and set $\h=\mathrm{Lie}(H)\otimes\C$ and 
$H_r=H\cap G_r$.
Define
$$
\Delta(h)=(\det(\Ad(h)\inv -1)|_{\g/\h}\quad (h\in H_r).
$$

We identify the universal enveloping algebra $\mathfrak U(\g)$ with the algebra of left-invariant differential operators on $G$. 
Let $\z$ be the center of  $\mathfrak U(\g)$.
Similarly we identify $S(\h)$ with the algebra of differential operators on $H$. Let $W$ be the Weyl group of $H$ in $G$, and 
let $\gamma:\z\rightarrow S(\h)^W$ be the Harish Chandra isomorphism.
Let $G_r$ be the open set of regular semisimple elements of $G$, and let $C^\infty$ be the smooth function on $G_r$. 

\begin{lem}
Suppose $f\in C^\infty(G_r)$ is a class function. Then
$$
(zf)(h)=|\Delta(h)|^{-\frac12}\gamma(z)(|\Delta|^{\frac12}f)(h)\quad (h\in H_r).
$$
\end{lem}

\begin{proof}
This statement can be found in the proof of \cite[Lemma 24]{HC1}. To be precise, write $h=\exp(H)$ with $H\in \g$.
Then there is a neighborhood $U$ of $H$ such that the the map $\exp:U\rightarrow U_G=\exp(H)$ is an analytic diffeomorphism
\cite[Section 10]{HC1}. Let $\xi$ be the analytic, non-zero function on $U$ of \cite[Section 10]{HC1}. 
If $\phi$ is a locally invariant $C^\infty$ function on $U$ (see \cite[Section 8]{HC3}) let $f_\phi(\exp X)=\xi(X)\inv \phi(X)$.
The map $\phi\rightarrow f_\phi$ is a bijection between the smooth locally invariant functions on $U$ and $U_G$.

According to the proof of \cite[Lemma 24]{HC1} it follows from 
\cite[Lemma 17]{HC1} and \cite[Theorem 2]{HC2} that
\begin{equation}
\label{1}
(zf_\phi)(h)=|\Delta(h)|^{-\frac12}\gamma(z)(|\Delta|^{\frac12}f_\phi)(h)\quad (h\in H_r).
\end{equation}

By the preceding discussion $f|_{U_G}$ is a locally invariant function, so $f_{U_G}=f_\phi$ for some locally invariant
function $\phi$ on $U$. The result follows.
\end{proof}


\section{Discussion}

Surely this result is known, but I could not find a clear statement and proof. In particular I do not see how \eqref{1}
follows from the cited results. Here are a few comments. All references are to \cite{HC1} unless otherwise noted. 

The references for the proof of \eqref{1} are Lemma 17 and 
 \cite[Theorem 2]{HC2}. I don't see how Lemma 17 applies, except as a step in the proof of Lemma 18, which {\it is} relevant. 
However Lemma 18 refers to the locally defined differential operator $\delta_a(z)$ (see Section 5). On the other hand 
\cite[Theorem 2]{HC2} refers to the differential operator $\beta(z)$ which is defined on all of $H_r$. 
I {\it think} that the proof is using the fact that if we write $h=ay$ with $y$ near the identity then
$\beta(z)(f)(ay)=\delta_a(z)(f)(ay)$. However I was unable to convince myself this is true.

Note that Lemma 13 says that
$$
\delta_a(z)(f)(h)=|\Delta_a(h)|^{-\frac12}\gamma(z)(|\Delta_a|^{\frac12}f)(h)
$$
where 
$$
\Delta_a(h)=\Delta(ah).
$$
This is very close to the statement needed for the result, except that it involves the function $\Delta_a$, not $\Delta$.
This has to do with the issue of defining the differential operator on $H_r$ using the local expression at each point.
It is worth pointing out that there are some fraught technical details here. For example see the proof of Lemma 23 and 
\cite[Lemma 25]{HC2}.
Finally, note that Harish Chandra does {\it not} cite Lemma 13 in his proof, but rather the more distantly related \cite[Theorem 2]{HC2}.

The conclusion is that I don't know how to fill in the details of the proof.






\bibliographystyle{amsalpha}
%\bibliography{/Users/kaletha/Work/TexMain/bibliography.bib}
\bibliography{bibliography.bib}

\end{document}

